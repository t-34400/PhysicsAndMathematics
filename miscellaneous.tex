\documentclass[]{article}
\usepackage{amsfonts} 
\usepackage{amsthm}
\usepackage{amsmath}
\usepackage{graphicx}
\usepackage{here}
\usepackage{hyperref}
\usepackage{tikz}

\theoremstyle{definition}
\newtheorem{definition}{Definition}[section]

\title{Miscellaneous}
\author{@t-34400}

\newcommand{\dif}{\mathrm{d}}
\newcommand{\deriv}[2]{\frac{\dif #1}{\dif #2}}
\newcommand{\parderiv}[2]{\frac{\partial #1}{\partial #2}}


\begin{document}
\maketitle
\tableofcontents

\section{Strain}
\definition
Let $\vec{u}$ be a desplacement field. The elements of \textbf{strain tensor} is defined as
\begin{equation}
    \varepsilon_{ij}\equiv\frac{1}{2}\left(\deriv{u_i}{x_j} - \deriv{u_j}{x_i}\right)
\end{equation}
\definition
The \textbf{principal strains} are defined as eigenvalues of strain tensor.

\subsection{Triangle mesh (2D strain)}
Let the infinitesimal triangular surface $OAB$ be deformed into an infinitesimal triangular surface $O^\prime A^\prime B^\prime$. 
\begin{figure}[H]
    \centering
    \includegraphics{./.latex-img/.latex/miscellaneous_triangle_mesh_2d_strain.pdf}
    \label{fig:infinitesimal_triangle_deformation_2d_strain}
\end{figure}
When considering strain, the rigid body transformation part can be ignored, so it can be assumed that $O$ and $O^\prime$ are the same point, and $O$, $A$ and $A^\prime$ are colinear, and $OAB$ and $O A^\prime B^\prime$ are in the same plane. 
Then we define the $x_0$-axis as the $OA$ direction and the $x_1$-axis so that the $x_0x_1$ plane contains the triangles $OAB$ and $OA^\prime B^\prime$.
In this case, the derivatives of the deformation vector are derived from the linear equations
\begin{equation}
    \vec{a}^\prime = \mathbb{U}\vec{a}, \qquad\vec{b}^\prime = \mathbb{U}\vec{b}
\end{equation}
where $\mathbb{U}_{ij} = \partial u_i/\partial x_j$, and become
\begin{equation}
    \begin{aligned}
        \parderiv{u_0}{x_0} = \frac{|\vec{a}^\prime|-|\vec{a}|}{|\vec{a}|},&\qquad\parderiv{u_0}{x_1} = \frac{b^\prime_0-(1 + \partial u_0/\partial x_0) b_0}{b_1}\\
        \parderiv{u_1}{x_0} = 0,&\qquad\parderiv{u_1}{x_1} = \frac{b^\prime_1 - b_1}{b_1}
    \end{aligned}
\end{equation}
The normal strains and shear strains are described as follows:
\begin{equation}
    \begin{aligned}
        \varepsilon_{00} &= \frac{|\vec{a}^\prime| - |\vec{a}|}{|\vec{a}|}\\
        \varepsilon_{11} &= \frac{b^\prime_1 - b_1}{b_1}\\
        \varepsilon_{10} = \varepsilon_{01} &= \frac{b^\prime_0 - (1+\varepsilon_{00})b_0}{2b_1}
    \end{aligned}
\end{equation}

The principal strains are defined as the eigenvalues of the strain tensor, so the principal strains are as follows:
\begin{equation}
    \varepsilon_{\mathrm{max}}, \varepsilon_{\mathrm{min}} = \frac{1}{2}\left(\varepsilon_{00} + \varepsilon_{11} \pm \sqrt{(\varepsilon_x + \varepsilon_y)^2 - 4(\varepsilon_{00}\varepsilon_{11}-\varepsilon_{01}\varepsilon_{10})}\right)    
\end{equation}

\subsubsection{Shader program}
Refer to \href{https://github.com/t-34400/PhysicsAndMathematics/tree/main/SampleCode/PrincipalStrainShader}{\textbf{SampleCode/PrincipalStrainShader}} for a sample code of shaders that calculates the 2D principal strains from the original and current vertex positions, and color meshes based on their principal strains.

\subsection{Triangle mesh (3D strain)}
Mathematically, a 2D object generally has a 2x2 strain tensor. 
However, Although 3D model meshes are treated as 2D planes, the objects they represent originally have a thickness in the third dimension. 
Therefore, considering the strain in the vertical direction of the surface by assuming that the mesh has a small thickness may allow for a more realistic calculation of strain.

Considering a infinitesimal triangle surface $OAB$ which the vertex normal of the vertex $O$ is parallel to the normal of this surface.
Let this surface $OAB$ be deformed into an infinitesimal triangular surface $O^\prime A^\prime B^\prime$.
Here, the vertical strain is assumed to be obtained from the displacement of each edge in the direction normal of the vertex normal of $O^\prime$.
Since the rigid body transformation part can be ignored, we can assume that the projection of the edge $O^\prime A^\prime$ onto a $x_0x_1$-plane is parallel to the edge $OA$ (let this direction be $x_0$-direction).
\begin{figure}[H]
    \centering
    \includegraphics{./.latex-img/.latex/miscellaneous_triangle_mesh_3d_strain.pdf}
    \label{fig:infinitesimal_triangle_deformation_3d_strain}
\end{figure}

In this case, we can perform the same discussion as in the 2D case regarding the derivatives of the displacement vector in the $x_0y_0$ plane 
by corresponding the projections $O^\prime A^{\prime\prime}, O^\prime B^{\prime\prime}$ of the edges $OA^\prime, OB^\prime$ onto the $x_0x_1$-plane to $\vec{a}^\prime, \vec{b}^\prime$:
\begin{equation}
    \begin{aligned}
        \parderiv{u_0}{x_0} = \frac{a^\prime_0-a_0}{a_0},&\qquad\parderiv{u_0}{x_1} = \frac{b^\prime_0-(1 + \partial u_0/\partial x_0) b_0}{b_1}\\
        \parderiv{u_1}{x_0} = 0,&\qquad\parderiv{u_1}{x_1} = \frac{b^\prime_1 - b_1}{b_1}
    \end{aligned}
\end{equation}
and other coordinates are derived as
\begin{equation}
    \begin{aligned}
        \parderiv{u_2}{u_0} &= \frac{a_2^\prime}{a_0}, \qquad \parderiv{u_2}{u_1} = \frac{b_2^\prime - \partial u_2/\partial x_0 b_0}{b_1}\\
        \parderiv{u_i}{x_2} &= 0            
    \end{aligned}
\end{equation}

The normal strains and shear strains are described as follows:
\begin{equation}
    \mathbb{\varepsilon} = 
    \begin{pmatrix}
        \frac{a_0^\prime-a_0}{a_0} & \frac{b_0^\prime-(1+\partial u_0/\partial x_0)b_0}{2b_0} & \frac{a_2^\prime}{2a_0} \\
        \frac{b_0^\prime-(1+\partial u_0/\partial x_0)b_0}{2b_0} & \frac{b_1^\prime - b_1}{b_1} & \frac{b_2^\prime - \partial u_2/\partial x_0 b_0}{2b_1}\\
        \frac{a_2^\prime}{2a_0} &  \frac{b_2^\prime - \partial u_2/\partial x_0 b_0}{2b_1} & 0
    \end{pmatrix}
\end{equation}

\end{document}