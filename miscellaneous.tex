\documentclass[]{article}
\usepackage{amsthm}
\usepackage{amsmath}
\usepackage{graphicx}
\usepackage{here}
\usepackage{hyperref}
\usepackage{tikz}

\theoremstyle{definition}
\newtheorem{definition}{Definition}[section]

\title{Miscellaneous}
\author{@t-34400}

\begin{document}
\maketitle
\tableofcontents

\section{Strain}
\definition{strain matrix}
\subsection{Triangle mesh (2D strain)}
Let the infinitesimal triangular surface $OAB$ be deformed into an infinitesimal triangular surface $O^\prime A^\prime B^\prime$. 
\begin{figure}[H]
    \centering
    \includegraphics{./.latex-img/.latex/miscellaneous_triangle_mesh_2d_strain.pdf}
    \label{fig:infinitesimal_triangle_deformation_2d_strain}
\end{figure}
When considering strain, the rigid body transformation part can be ignored, so it can be assumed that $O$ and $O^\prime$ are the same point, and $O$, $A$ and $A^\prime$ are colinear, and $OAB$ and $O A^\prime B^\prime$ are in the same plane. 
Then we define the $x_0$-axis as the $OA$ direction and the $x_1$-axis so that the $x_0x_1$ plane contains the triangles $OAB$ and $OA^\prime B^\prime$.
In this case, the normal strains and shear strains are described as follows:
\begin{equation}
    \begin{aligned}
        \varepsilon_{00} &= \frac{|\vec{a}^\prime| - |\vec{a}|}{|\vec{a}|}\\
        \varepsilon_{11} &= \frac{b^\prime_1 - b_1}{b_1}\\
        \varepsilon_{10} = \varepsilon_{01} &= \frac{b^\prime_0 - (1+\varepsilon_{00})b_0}{2b_1}
    \end{aligned}
\end{equation}

The principal strains are defined as the eigenvalues of the strain tensor, so the principal strains are as follows:
\begin{equation}
    \varepsilon_{\mathrm{max}}, \varepsilon_{\mathrm{min}} = \frac{1}{2}\left(\varepsilon_{00} + \varepsilon_{11} \pm \sqrt{(\varepsilon_x + \varepsilon_y)^2 - 4(\varepsilon_{00}\varepsilon_{11}-\varepsilon_{01}\varepsilon_{10})}\right)    
\end{equation}

\subsubsection{Shader program}
Please refer to \href{https://github.com/t-34400/PhysicsAndMathematics/tree/main/SampleCode/PrincipalStrainShader}{\textbf{SampleCode/PrincipalStrainShader}} for a sample code of shaders that calculates the 2D principal strains from the original and current vertex positions, and color meshes based on their principal strains.

\end{document}