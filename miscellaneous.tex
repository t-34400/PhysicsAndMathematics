\documentclass[]{article}
\usepackage{amsthm}
\usepackage{amsmath}
\usepackage{hyperref}
\usepackage{tikz}

\theoremstyle{definition}
\newtheorem{definition}{Definition}[section]

\title{Miscellaneous}
\author{@t-34400}

\begin{document}
\maketitle
\tableofcontents

\section{Strain}
\definition{strain matrix}
\subsection{Trianlge mesh (2D strain)}
Let the infinitesimal triangular surface $OAB$ be deformed into an infinitesimal triangular surface $OA^\prime B^\prime$. When considering strain, the rigid body transformation part can be ignored, so it can be assumed that $O$, $A$ and $A^\prime$ are colinear and $OAB$ and $OA^\prime B^\prime$ are in the same plane. 

In this case, the normal strains are described as follows:
\begin{equation}
    \begin{aligned}
        \varepsilon_{00} &= \frac{OA^\prime - OA}{OA}\\
        \varepsilon_{11} &= \frac{OB^\prime - OB}{OB}
    \end{aligned}
\end{equation}
and the shear strain is described as follows:
\begin{equation}
    \gamma_{xy} = \gamma_{yx} = \frac{1}{2}\tan\left(\angle AOB - \angle A^\prime OB^\prime\right)    
\end{equation}

The principal strains are defined as the eigenvalues of the strain tensor:
\begin{equation}
    \begin{pmatrix}
        \varepsilon_x && \gamma_{xy} \\
        \gamma_{yx} && \varepsilon_y
    \end{pmatrix}        
\end{equation}
so the principal strains are as follows:
\begin{equation}
    \varepsilon_{\mathrm{max}}, \varepsilon_{\mathrm{min}} = \frac{1}{2}\left(\varepsilon_x + \varepsilon_y \pm \sqrt{(\varepsilon_x + \varepsilon_y)^2 + 4(\varepsilon_x\varepsilon_y-\gamma_{xy}\gamma_{yx})}\right)    
\end{equation}

\subsubsection{Shader program}
Please refer to \href{https://github.com/t-34400/PhysicsAndMathematics/SampleCode/PrincipalStrainShader}{\textbf{SampleCode/PrincipalStrainShader}} for a sample code of shaders that calculates the 2D principal strains from the original and current vertex positions, and colors them based on their magnitude.

\end{document}